\documentclass{article}
\usepackage{geometry}
\usepackage{amssymb}
\usepackage{amsmath}
\usepackage{xcolor}
\usepackage{booktabs}
\usepackage{graphicx}
\usepackage{float}
\usepackage{caption}
\usepackage[italian]{babel}
\usepackage{xcolor}
\usepackage{listings}

\definecolor{backcolor}{rgb}{0.95, 0.95, 0.95}
\definecolor{codegreen}{rgb}{0.25, 0.45, 0.25}

\usepackage{tikz}
\usetikzlibrary{shapes.geometric, arrows}

\tikzstyle{startstop} = [rectangle, rounded corners, minimum width=3cm, minimum height=1cm,text centered, draw=black, fill=red!30]
\tikzstyle{passage} = [rectangle, minimum width=3cm, minimum height=1.5cm, text centered, draw=black, fill=orange!30]
\tikzstyle{operations} = [rectangle, minimum width=3cm, minimum height=1cm, text centered, draw=black, fill=green!30]
\tikzstyle{foo} = [rectangle, minimum width=3cm, minimum height=1cm, text centered, draw=black, fill=violet!30]
\tikzstyle{arrow} = [thick,->,>=stealth]

% Define Python style
\lstdefinestyle{Pythonstyle}{
    language=Python,
	basicstyle=\ttfamily,
	tabsize=4,
    backgroundcolor=\color{backcolor},
    commentstyle=\color{codegreen},
    keywordstyle=\color{magenta},
    identifierstyle=\color{violet},
    stringstyle=\color{green!50!black},
    showstringspaces=false,
	showtabs=false,
    numbers=left,
    numberstyle=\tiny,
    numbersep=5pt,
    breaklines=true,
    extendedchars=true,
    linewidth=\textwidth,
	morekeywords={*,...}
}

\setcounter{page}{0}

\geometry{
	left=.6in,  
	right=.6in, 
	top=.6in,   
	bottom=.6in,
}


\title{Copilota RCQF}
\author{Alessio Cimma}

\begin{document}

\maketitle

\begin{center}
	\includegraphics*[width=.99\linewidth]{../images/logo.png}
\end{center}

\newpage
\tableofcontents

% \newpage
% \section{Grafici / Plot}
% \subsection{Overview}
% \subsection{Implementazioni}
% \subsection{Integrazione}
% \newpage
% \section{Statistica}
% \subsection{Overview}
% \subsection{Implementazioni}
% \subsection{Integrazione}
% \newpage
% \section{Excel / Data entry}
% \subsection{Overview}
% \subsection{Implementazioni}
% \subsection{Integrazione}

\newpage
\section{3D orientation}
\subsection{Pipeline di renderizzazione}

\begin{tikzpicture}[node distance=2.5cm]
	
	\node (start) [startstop] {Start};
	
	\node (passage1) [passage, below of=start] {World Space};
	\node (matrice1) [operations, right of=passage1, xshift=1.5cm] {Attributi Modello};
	\node (foo1) [foo, right of=matrice1, xshift=2cm] {scalotrasla()};
	\node (foo1_2) [foo, right of=foo1, xshift=2cm] {rotx(), roty(), rotz()};
	
	\node (passage2) [passage, below of=passage1] {Camera Space};
	\node (matrice2) [operations, below of=matrice1] {RotoTraslazione camera};
	\node (foo2) [foo, right of=matrice2, xshift=1.5cm] {camera\_world()};
	\node (foo2_2) [foo, right of=foo2, xshift=1cm] {frustrum()};
	\node (foo2_3) [foo, right of=foo2_2, xshift=1cm] {proiezione()};
	
	\node (passage3) [passage, below of=passage2] {Screen Space};
	\node (matrice3) [operations, below of=matrice2] {Cambio di base};
	\node (foo3) [foo, right of=matrice3, xshift=3cm] {screen\_world()};
	\node (foo3_2) [foo, right of=foo3, xshift=3cm] {centra\_schermo()};
	
	\node (stop) [startstop, below of=passage3] {Stop};
	
	\draw [arrow] (start) -- (passage1);
	\draw [arrow] (passage1) -- (passage2);
	\draw [arrow] (passage2) -- (passage3);
	\draw [arrow] (passage3) -- (stop);
	
	\draw [arrow] (passage1) -- (matrice1);
	\draw [arrow] (passage2) -- (matrice2);
	\draw [arrow] (passage3) -- (matrice3);
	
	\draw [arrow] (matrice1) -- (foo1);
	\draw [arrow] (foo1) -- (foo1_2);

	\draw [arrow] (matrice2) -- (foo2);
	\draw [arrow] (foo2) -- (foo3);
	\draw [arrow] (foo2_2) -- (foo2_3);
	\draw [arrow] (foo2_3) -- (foo3_2);
	
	\draw [arrow] (matrice3) -- (foo3);
	\draw [arrow] (foo3) -- (foo2_2);
	
\end{tikzpicture}

\hrulefill
\subsection{Pipeline di aggiornamento}
\begin{tikzpicture}[node distance=2.5cm]
	
	\node (start) [startstop] {Input};
	
	\node (passage1) [passage, below of=start] {mouse delta XY};
	\node (matrice1) [operations, right of=passage1, xshift=1.5cm] {rototrasl. separate};
	\node (foo1) [foo, above of=matrice1, xshift=2cm] {aggiorna\_attributi1()};
	\node (foo1_2) [foo, below of=matrice1, xshift=2cm] {aggiorna\_attributi2()};
	
	\node (passage2) [passage, below of=passage2] {keyboard};
	\node (matrice2) [operations, right of=passage2, xshift=1.5cm] {aggiornamento attributi};
	
	\draw [arrow] (start) -- (passage1);
	\draw [arrow] (passage1) -- (passage2);
	
	\draw [arrow] (passage1) -- (matrice1);
	\draw [arrow] (passage2) -- (matrice2);
	
	\draw [arrow] (matrice1) -- (foo1);
	\draw [arrow] (matrice1) -- (foo1_2);
	
\end{tikzpicture}

\newpage

\subsection{Teoria della pipeline usata}
\paragraph{Modello:} La geometria del modello solitamente è descritta e contenuta dentro ad un file \texttt{model.obj} c'entrata su un'origine a $XYZ = \{0, 0, 0\}$. Quando questo modello viene caricato, viene inserito in un'apposita classe che contiene anche attributi come posizione, scala e rotazione. Queste verranno applicate ad ogni refresh della pagina con il seguente ed immutabile ordine:
\begin{itemize}
	\item Rotation X
	\item Rotation Y
	\item Rotation Z
	\item Scale and Traslation
\end{itemize}
%
Questo passaggio dev'essere il primo ad essere eseguito, in quanto posiziona tutti i modelli in uno spazio globale in relazione uno con l'altro.

\paragraph{Camera:} Usando una camera e non un punto di vista fisso, dobbiamo spostarci dallo spazio globale a quello della camera, per farlo usiamo la relativa matrice che trasla e ruota la scena. (Immagino che in futuro aggiungerò la possibilità di avere più camere tra cui scegliere; durante la pipeline, basterà switchare la camera attiva). Una volta compiuto questo passaggio, nel caso ci trovassimo in modalità prospettiva, sarà necessario applicare \texttt{frustrum()} e \texttt{proiezione()}, che appunto generano l'effetto di proiezione dato un certo FOV.

\paragraph{Schermo:} In questo programma noi usiamo un sistema di riferimento per cui:
\begin{itemize}
	\item la destra locale è X - la destra sullo schermo è X
	\item l'alto locale è Z - l'alto sullo schermo è -Y
	\item il profondo locale è Y - il profondo sullo schermo è Z
\end{itemize} 
%
Questo ci porta a dover cambiare la base per avere tutto orientato correttamente. Infine verrà applicata una scala per portarci da uno spazio normalizzato (-1, 1) ad uno spazio nella scala della dimensione dello schermo.

\paragraph{Aggiornamento degli attributi:} Nel caso dei modelli, basterà aggiornare la nuova pos / rot / scala.
Nel caso della camera il discorso si fa più complesso. Dobbiamo suddividere il discorso in due parti:
\begin{itemize}
	\item Orientamento: aggiornato seguendo la sequenza di rotazioni di Eulero (XYZ)
	\item Posizione: la rotazione attorno al primo asse (Y) è facilmente applicabile, il problema compare quando si cerca di applicare la seconda rotazione (X). Infatti ora la dovremo applicare lungo la X locale, che è diversa dalla X globale. Non possiamo quindi applicare \texttt{rotx()}, ma \texttt{rot\_ax()} attorno all'asse locale X
\end{itemize}

\newpage

\subsection{Codice delle varie funzioni}
\begin{lstlisting}[style=Pythonstyle, caption={Matrici di Rotazione (3 con assi globali, 1 con asse custom)}]
@staticmethod
def rotx(ang: float) -> np.ndarray[np.ndarray[float]]:
	return np.array([
		[1, 0, 0, 0],
		[0, np.cos(ang), np.sin(ang), 0],
		[0, -np.sin(ang), np.cos(ang), 0],
		[0, 0, 0, 1]
	])

@staticmethod
def roty(ang: float) -> np.ndarray[np.ndarray[float]]:
	return np.array([
		[np.cos(ang), 0, np.sin(ang), 0],
		[0, 1, 0, 0],
		[-np.sin(ang), 0, np.cos(ang), 0],
		[0, 0, 0, 1]
	])

@staticmethod
def rotz(ang: float) -> np.ndarray[np.ndarray[float]]:
	return np.array([
		[np.cos(ang), np.sin(ang), 0, 0],
		[-np.sin(ang), np.cos(ang), 0, 0],
		[0, 0, 1, 0],
		[0, 0, 0, 1]
	])
	
@staticmethod
def rot_ax(axis: np.ndarray[float], ang: float) -> np.ndarray[np.ndarray[float]]:
	K = np.array([
		[0, -axis[2], axis[1], 0],
		[axis[2], 0, -axis[0], 0],
		[-axis[1], axis[0], 0, 0],
		[0, 0, 0, 1]
	])

	return np.eye(4) + np.sin(ang) * K + (1 - np.cos(ang)) * np.dot(K, K)
\end{lstlisting}

\begin{lstlisting}[style=Pythonstyle, caption={Matrice di ScaloTraslazione}]
@staticmethod
def scalotrasla(obj):
	return np.array([
		[obj.sx, 0, 0, 0],
		[0, obj.sy, 0, 0],
		[0, 0, obj.sz, 0],
		[obj.x, obj.y, obj.z, 1]
	])
\end{lstlisting}

\newpage

\begin{lstlisting}[style=Pythonstyle, caption={Matrice di cambio sistema di riferimento (camera)}]
@staticmethod
def camera_world(camera: np.ndarray) -> np.ndarray[np.ndarray[float]]:
	return np.array(
		[[1, 0, 0, 0],
			[0, 1, 0, 0],
			[0, 0, 1, 0],
			[-camera.pos[0], -camera.pos[1], -camera.pos[2], 1]]
	) @ np.array(
		[[camera.rig[0], camera.dir[0], camera.ups[0], 0],
			[camera.rig[1], camera.dir[1], camera.ups[1], 0],
			[camera.rig[2], camera.dir[2], camera.ups[2], 0],
			[0, 0, 0, 1]]
	)
\end{lstlisting}

\begin{lstlisting}[style=Pythonstyle, caption={Matrici di applicazione prospettiva}]
@staticmethod
def frustrum(W: int, H: int, h_fov: float = np.pi / 6) -> np.ndarray[np.ndarray[float]]:
	v_fov = h_fov * H / W
	ori = np.tan(h_fov / 2)
	ver = np.tan(v_fov / 2)
	return np.array([
		[1 / ori, 0, 0, 0],
		[0, 1 / ver, 0, 0],
		[0, 0, 1, 1],
		[0, 0, 0, 0]
	])

@staticmethod
def proiezione(vertici: np.ndarray[np.ndarray[float]]) -> np.ndarray[np.ndarray[float]]:
	distanze = vertici[:, -1]
	indici_dist_neg = np.where(distanze < 0)

	ris = vertici / vertici[:, -1].reshape(-1, 1)
	ris[(ris < -12) | (ris > 12)] = 2
	ris[indici_dist_neg] = np.array([2, 2, 2, 1])

	return ris
\end{lstlisting}


\begin{lstlisting}[style=Pythonstyle, caption={Matrice di cambio sistema di riferimento (schermo)}]
	@staticmethod
	def screen_world() -> np.ndarray[np.ndarray[float]]:
	return np.array([
		[1,0,0,0],
		[0,0,1,0],
		[0,-1,0,0],
		[0,0,0,1]
		])
\end{lstlisting}
	
\begin{lstlisting}[style=Pythonstyle, caption={Matrice di adattamento spazio normalizzato a dimensione dello schermo in px}]
@staticmethod
def centra_schermo(W, H):
return np.array([
	[W/2, 0, 0, 0],
	[0, H/2, 0, 0],
	[0, 0, 1, 0],
	[W/2, H/2, 0, 1]
	])
\end{lstlisting}

\newpage

\begin{lstlisting}[style=Pythonstyle, caption={prima e seconda parte dell'aggiornamento degli attributi della camera}]
def rotazione_camera(self) -> None:
	'''
	Applico le rotazioni in ordine Eulero XYZ ai vari vettori di orientamento della camera
	'''
	self.rig = self.rig_o @ Mate.rotx(self.becche)
	self.ups = self.ups_o @ Mate.rotx(self.becche)
	self.dir = self.dir_o @ Mate.rotx(self.becche)

	self.rig = self.rig @ Mate.roty(self.rollio)
	self.dir = self.dir @ Mate.roty(self.rollio)
	self.ups = self.ups @ Mate.roty(self.rollio)

	self.rig = self.rig @ Mate.rotz(self.imbard)
	self.ups = self.ups @ Mate.rotz(self.imbard)
	self.dir = self.dir @ Mate.rotz(self.imbard)

	'------------------------------------------------------------------'

	self.pos -= self.focus
	self.pos = self.pos @ Mate.rotz(- self.delta_imbard)
	self.pos = self.pos @ Mate.rot_ax(self.rig, self.delta_becche)
	self.pos += self.focus
\end{lstlisting}

\newpage

\section{Camera sync}
Dal momento che usiamo le matrici per generare la prospettiva nel caso del wireframe e generiamo dei raggi per ricreare lo stesso effetto nel raytracer, è necessario trovare un sistema che unifica questi metodi per riottenere la stessa immagine.
\\
Ci baseremo sul fornire ad entrambi i sistemi una variabile \texttt{fov} che indicherà l'apertura focale in radianti.
\subsection{Approccio matriciale}
\begin{lstlisting}[style=Pythonstyle, caption={Matrici di applicazione prospettiva}]
@staticmethod
def frustrum(W: int, H: int, h_fov: float=np.pi/6) -> np.ndarray[float]:
	v_fov = h_fov * H / W
	ori = np.tan(h_fov / 2)
	ver = np.tan(v_fov / 2)
	return np.array([
		[1 / ori, 0, 0, 0],
		[0, 1 / ver, 0, 0],
		[0, 0, 1, 1],
		[0, 0, 0, 0]
	])
\end{lstlisting}
%
\paragraph{Spigazione:}
Con questo sistema noi possiamo calcolare la divergenza della geometria sfruttando il funzionamento della prospettiva stessa, ovvero che i punti tendono a seguire uno spostamento di $\cot(\text{fov})$, il quale successivamente verrà diviso per la propria distanza.

\subsection{Approccio generazione raggi}

\begin{lstlisting}[style=Pythonstyle, caption={Calcolo della direzione del raggio generato in base al fov}]
def direzione(i, j) -> float:
	return camera.dir[:3] * (1 / np.tan(camera.fov / 2)) + (2 * (i / wo) - 1) * camera.rig[:3] + (2 * (j / ho) - 1) * (- camera.ups[:3]) / (wo/ho)
\end{lstlisting}
%
\paragraph{Spigazione:}
Dal momento che della camera noi conosciamo la direzione, la sua destra e il suo alto, possiamo calcolare di quanto dovrebbe estendersi il versore direzione affinchè sommato con il versore destra e alto, ci restituisca la direzione corretta. Nel codice manca la normalizzazione del risultato (eseguita più avanti). 
La necessità di dividere per 2 l'angolo del \texttt{fov} è dovuta alla scala usata. Infatti nel calcolo matriciale la base del triangolo isoscele che si forma considerando il vertice coincidente con la posizione della camera, avrà dimensione 1. Nel Case invece dei raggi generati, i versori saranno $\pm$ 1, dandoci un $\Delta$ tot = 2, e che quindi corrisponde a scala doppia.

\subsection{Risultati}

\begin{center}
    \begin{minipage}{0.49\textwidth}
        \includegraphics*[width=\linewidth]{../images/rendered.png}
    \end{minipage}
    \hfill
    \begin{minipage}{0.49\textwidth}
        \includegraphics*[width=\linewidth]{../images/wireframe.png}
    \end{minipage}
\end{center}

\newpage
\section{Ray - Sphere intersection}

Il problema principale risiede nel calcolare il risultato dell'equazione per ottenere l'intersezione (minima e massima).
Per risolverla dobbiamo risolvere l'equazione che mette in relazione: origine del raggio, direzione del raggio, posizione della sfera e raggio della sfera.

$$| ray.ori + ray.dir * t | ^ 2 = | sphere.pos + sphere.radius | ^ 2$$

Sappiamo che risulta tutto più comodo se la sfera è centrata sull'origine, possiamo quindi fare una trasformazione e chiamare $oc$ la differenza tra $\vec{ray.ori} - \vec{sphere.ori}$.

$$| oc + ray.dir * t | ^ 2 = | sphere.radius | ^ 2$$

$$oc^2 + (ray.dir * t) ^ 2 + 2*oc*ray.dir*t = sphere.radius ^ 2$$

$$ray.dir^2 * t^2 + 2*oc*ray.dir*t + oc^2 - sphere.radius ^ 2 = 0$$

Se volessimo risolvere per t, otterremmo un'equazione di secondo grado di forma: $ax^2 + bx + c = 0$, dove $a = ray.dir^2$, il quale essendo un versore sarà uguale a 1; $b = 2*oc*ray.dir$; $c = oc^2 - sphere.radius^2$. 
\\
Da qui risolviamo il discriminante e otteniamo i possibili risultati.
\paragraph{Discussione risultati:} Dobbiamo sempre considerare il risultato minimo positivo per eseguire la corretta intersezione. Attenzione che nel caso del vetro bisognerà considerare quello più lontano in certi casi, qualora si volesserò simulare trasmissioni ecc.
\paragraph{Spiegazioni del codice:} Eseguire il prodotto scalare di un vettore per se stesso, ci darà il valore del suo modulo. La presenza del test $> 0.001$ invece, è dovuto per evitare errori di appossimazioni in cui l'origine del raggio successivo è di poco al di sotto della superficie, risultando quindi in una collisione con se stesso.

\begin{lstlisting}[style=Pythonstyle, caption={Codice intersezione raggio - sfera}]
def collisione_sphere(self, ray, record: Record):
	oc = ray.pos - self.pos;
	
	a = 1.0;
	b = np.dot(oc, ray.dir) * 2.0;
	c = np.dot(oc, oc) - (self.radius) ** 2;

	discriminante = b ** 2 - 4.0 * a * c;
	
	if discriminante >= 0.0:
		sqrt_discr = discriminante ** 0.5
		
		delta_min = (- b - sqrt_discr) / (2.0*a);
		delta_max = (- b + sqrt_discr) / (2.0*a);
		
		if delta_max > 0.001:
			t = delta_max
			if delta_min > 0.001 and delta_min < delta_max:
				t = delta_min

			if t < record.distanza:
				record.colpito = True;
				record.distanza = t;
				record.indice_oggetti_prox = self.indice;
				record.punto_colpito = self.punto_colpito(ray, record.distanza);
				record.norma_colpito = self.normale_colpita(record.punto_colpito);
				record.materiale = self.materiale;
		
	return record
\end{lstlisting}

\newpage
\section{Ray - Triangle intersection}

Questo sarà l'algoritmo su cui si baserà la maggior parte dei test. Le sfere verranno usate per la luce, i triangoli invece per le mesh 

% \newpage
% \section{Raytracer}
% \subsection{Overview}
% \subsection{Implementazioni}
% \subsection{Integrazione}
% \subsection{Guideline}

% \begin{itemize}
% \item \underline{Cambio scena:} Bisognerà poter switchare da plot a raytracer
% \item \underline{Schermo per visualizzare un array:} Adattarsi alla dimensione, controllare performance, aggiungere controlli come bottoni di accensione, refresh, noise, dimensione e controlli generici
% \item \underline{Controlli di zoom:} Provare con zoom in, out, adattamento schermo, pan.
% \item \underline{Wireframe:} Aggiungere wireframe e pointcloud.
% \item \underline{Mesh:} Creare classe che contenga: punti, link, attributi
% \item \underline{Camera:} Creare classe Camera che regga il movimento, orbit pan e info sulla prospettiva
% \item \underline{UI:} Aggiungere UI per stare dietro ai parametri richiesti 
% \item \underline{Add / Delete:} Poter gestire la scena aggiungendo e rimuovendo oggetti e controllare i loro attributi
% \item \underline{Rasterization:} Implementare sistema di renderizzazione per pixel  
% \item \underline{Affine vs Perspective:} Sistemare Texture Sampler in cui andiamo a dividere per la profondità per mappare i valori e li rimoltiplichiamo per ottenere i veri valori
% \item \underline{Raytracer:} Impostare un sistema di lancio raggi. test normal, direzione, sphere mapping.
% \item \underline{Sfere:} 
% \item \underline{Luce:} 
% \item \underline{Triangoli:} 
% \item \underline{AO:} 
% \item \underline{Textures:} 
% \end{itemize}

% \newpage
% \section{Chimica}
% \subsection{Overview}
% \subsection{Implementazioni}
% \subsection{Integrazione}
% \newpage
% \section{Meccanica quantistica}
% \subsection{Overview}
% \subsection{Implementazioni}
% \subsection{Integrazione}
% \newpage
% \section{Cristallografia}
% \subsection{Overview}
% \subsection{Implementazioni}
% \subsection{Integrazione}
% \newpage
% \section{Tavola periodica}
% \subsection{Overview}
% \subsection{Implementazioni}
% \subsection{Integrazione}
% \newpage
% \section{Timer}
% \subsection{Overview}
% \subsection{Implementazioni}
% \subsection{Integrazione}
% \newpage
% \section{Appunti}
% \subsection{Overview}
% \subsection{Implementazioni}
% \subsection{Integrazione}
% \newpage
% \section{Salvataggio / Impostazioni}
% \subsection{Overview}
% \subsection{Implementazioni}
% \subsection{Integrazione}
% \newpage
% \section{UI}
% \subsection{Integrazione}
% \subsection{Funzionalità}
% \newpage
% \section{TODO}
% \newpage
% \section{Cool projects}
% \subsection{Particle simulator}
% \subsection{AI for simple physics}
% \subsection{Terrain simulator}
% \subsection{Sudoku solver (WFC)}
% \subsection{Integrazione Bonsai}
% \subsection{TODO / progress tracker}
% \subsection{Music}


\end{document}